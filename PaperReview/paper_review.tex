\documentclass[conference]{IEEEtran}
%\IEEEoverridecommandlockouts
% The preceding line is only needed to identify funding in the first footnote. If that is unneeded, please comment it out.
\usepackage{cite}
\usepackage{amsmath,amssymb,amsfonts}
\usepackage{algorithmic}
\usepackage{graphicx}
\usepackage{textcomp}
\usepackage{xcolor}
\def\BibTeX{{\rm B\kern-.05em{\sc i\kern-.025em b}\kern-.08em
    T\kern-.1667em\lower.7ex\hbox{E}\kern-.125emX}}
\begin{document}

\title{Paper Review 8DM50 \\ Multi-Contrast Super-Resolution MRI Through a
  Progressive Network \cite{b1} }

\author{\IEEEauthorblockN{Sander Derwig}
\IEEEauthorblockA{\textit{Computational Biology} \\
\textit{Eindhoven University of Technology}\\
s0901783 \\
s.derwig@student.tue.nl}
\and
\IEEEauthorblockN{Manon van Erp}
\IEEEauthorblockA{\textit{IMAG/e} \\
\textit{Eindhoven University of Technology}\\
s0943658 \\
m.v.erp@student.tue.nl}
\and
\IEEEauthorblockN{Rutger Hendrix}
\IEEEauthorblockA{\textit{IMAG/e} \\
\textit{Eindhoven University of Technology}\\
s0902650 \\
r.hendrix@student.tue.nl}
\and
\IEEEauthorblockN{Dirk Loeffen}
\IEEEauthorblockA{\textit{IMAG/e} \\
\textit{Eindhoven University of Technology} \\
s1252399 \\
d.w.m.loeffen@student.tue.nl}
\and
\IEEEauthorblockN{Djennifer Madzia-Madzou}
\IEEEauthorblockA{\textit{Image Sciences Institute} \\
\textit{Utrecht University}\\
s1526960 \\
d.madziamazou@student.tue.nl}
}

\maketitle

%\begin{abstract}

%\end{abstract}

%\begin{IEEEkeywords}
%component, formatting, style, styling, insert
%\end{IEEEkeywords}

\section{Application Domain}
One of the limitations of the resolution of magnetic resonance (MR) imaging is
acquisition time. Acquisition time has to be limited in order for MR imaging
remain convenient for patients. To increase resolution without increasing
acquisition time, one could try one of several super-resolution (SR) techniques.
A more recent technique to reach super-resolution is deep learning. The reviewed
paper demonstrates two super resolution networks. 1) A one-level non-progressive
neural network and 2) two-level progressive neural network.

\section{Methods}
To train the neural networks they have to be fed low resolution (LR) images. The
(LR) images are acquired by down-sampling and zero-filling the high resolution
(HR) images in Fourier space. By doing so in Fourier space, the image size is
unchanged but the image quality is degraded.

To reach SR, two networks are trained. Both networks are a general adversarial
network (GAN), thus they consist of a generator and a discriminator. The
generator trains the SR, whereas the discriminator trains to qualify images as
real or fake. The difference between the two networks lays in the generator. In
the first network, the generator consists of an encoder, a decoder and a feature
extraction network that is applied to other MR sequences. The features extracted
from the other MR sequences are then fed into the network between the encoder
and the decoder. The second network, applies the encoder-decoder system two
times. In the second network the features from the other MR sequences are also
fed into the network twice, once in each encoder-decoder system.

In the reviewed paper the following four loss functions are used:

\begin{enumerate}
    \item The adversarial loss in the generative adversarial network framework is used to train the generator.
    \item The Mean-Squared Error evaluates the difference between the output of
      the generator and the corresponding ground truth at pixel-wise level. It
      can greatly improve the signal-to-noise ratio of generated images.
    \item The perceptual loss overcomes the problem that some details may be
      lost due to over-smoothed SR results. The perceptual loss recovers more
      details by measuring image similarity in a high-level feature space.
    \item The texture matching loss contributes to generate an image with great
      similarity between the output of the generator and the ground truth by
      statistically matching extracted features.
\end{enumerate}

The image quality evaluation metrics that are used in this paper are structural
similarity (SSIM), peak signal-to-noise ratio (PSNR) and information fidelity
criterion (IFC). The SSIM is a fidelity metric which compares the terms of
luminance, contrast and structural similarity. The second fidelity measure PSNR
relates to the MSE and can be expressed as the ratio between the square of the
maximum value and the MSE value. IFC measures image quality based on natural
scene statistics. IFC utilizes the mutual information between the reference and
a distorted image to evaluate image quality.

\section{Discussion and Recommendations}
One common problem in training a GAN is that it is
highly unstable. This is inherently the case since often the generator and the
discriminator are trained simultaneously while competing against each other. In
the aforementioned paper the discriminator was trained four times before the
generator was trained once. This might help stabilize the model.

Another strong point of the methodology is the use of multiple loss functions
and regularization factors. Each loss function has its own (dis)advantages. By
using the four loss functions, the power of each one is combined and results in
a generally better applicable loss function for this application.

Also two distorted images with the same MSE may have very different types of
errors, some of which are much more visible than others. One error might be
preferred over the other, which is now not addressed with the use of MSE.

The perceptual loss part in the objective function measures image similarity in
a high-level feature space. The pretrained VGG16 model on ImageNet is used to
extract features from a given image. However these images are normal images of
for example cars and so are not biomedical images. In our opinion extracting
features and training a model on biomedical images would be more reliable.
\newpage
PSNR is no longer regarded as a reliable indicator of image quality degradation,
and so is a weak point \cite{b2}. Although a higher PSNR generally indicates that the
reconstruction is of higher quality, in some cases it may not. Bear in mind that
none of these objective measures are particularly good at predicting human
visual response to image quality. Sometimes PSNRs vary wildly between two almost
indistinguishable images. SSIM is recommended. So since this is already used we
propose just skipping PSNR.

The Cognitive Interaction Problem. It is widely known that cognitive
understanding and interactive visual processing (e.g., eye movements) influence
the perceived quality of images. For example, a human observer will give
different quality scores to the same image if given different instructions
\cite{b2}.

\begin{thebibliography}{00}
\bibitem{b1} Lyu, Q., Shan, H., Steber, C., Helis, C., Whitlow, C. T., Chan, M.,
  & Wang, G. (2020). Multi-contrast super-resolution mri through a progressive
  network. IEEE Transactions on Medical Imaging.
\bibitem{b2} Wang, Z., Bovik, A. C., Sheikh, H. R., & Simoncelli, E. P. (2004).
  Image quality assessment: from error visibility to structural similarity. IEEE
  transactions on image processing, 13(4), 600-612.

\end{thebibliography}

\end{document}
